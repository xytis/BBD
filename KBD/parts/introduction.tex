
\section{Įvadas}

Pastaruoju metu itin sparčiai besivystanti organinių saulės elementų ir šviesos diodų pramonė sukuria poreikį naujų organinių medžiagų kūrimui ir jų tyrimui. Akivaizdu, jog kyla poreikis turėti tikslią ir patogią metodiką svarbiausioms medžiagos charakteristikoms nustatyti. Šiuo metu dėl savo matavimo įrangos paprastumo svarbia bandinių analizės metodika tapo krūvininkų ištraukimas tiesiškai kylančia įtampa (CELIV, \cite{juška:4946}). Duomenų analizės metu iš CELIV kinetikose esančio srovės maksimumo padėties galima suskaičiuoti krūvininkų judrį \cite{juška:4946}, tačiau buvo pastebėta, jog maksimumo padėtį keičia krūvininkų tankis,  rekombinacija ir elektrinis laukas \cite{bange:035209, lorrmann:113705}. Visai neseniai atkreiptas dėmesys ir į šviesos sugerties profilio įtaką gaunamiems rezultatams (photo-CELIV, \cite{juška:155202}). Pastarųjų tyrimų metu buvo iškeltos hipotezės leidžiančios nustatyti ar bandinyje vyrauja Lanževeno bimolekulinė rekombinacija \cite{juška:155202}. Dabar tyrėjai atkreipė dėmesį į difuzijos įtaką bendram pernašos mechanizmui.
Žinoma, jog difuzija keičia CELIV kinetikos pavidalą dėl įnešamos difuzinės srovės komponentės. Tačiau manoma, jog realaus eksperimento metu krūvininkų difuzija perskirsto krūvininkus taip, kad erdvinis pasiskirstymas paveikia rekombinacijos spartą ir veda prie klaidingo medžiagos parametrų nustatymo naudojantis klasikinėmis CELIV formulėmis  \cite{juška:4946}.Taigi reikia praplėsti dabar naudojamą modelį \cite{juška:155202} ir įtraukti difuzijos reiškinį.
Tačiau į CELIV aprašančias lygtis įvedus difuzijos narį susidaro sunkiai analitiškai apskaičiuojama integro-diferencialinė lygtis, kurią nutarta modeliuoti skaitmeniškai, ir taip stebėti sistemos elgseną.

Darbo tikslas -- skaitmeniškai modeliuojant nustatyti difuzijos įtaką krūvininkų pernašai photo-CELIV atveju, esant skirtingiems krūvininkų pradiniams pasiskirstymams. Nustatyti sąlygas, kurioms esant difuzija gali turėti įtakos krūvininkų rekombinacijos spartos matavimų rezultatus.