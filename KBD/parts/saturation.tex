 
\section{Sunkumai skaičiuojant kinetikų sotį}
\label{app:rekombinacija}

\begin{figure}[h]
  \centering
\subfloat[Be difuzijos]{\includegraphics[width=0.5\textwidth]{./media/img/bam_nodiff.png}}
\subfloat[Su difuzija]{\includegraphics[width=0.5\textwidth]{./media/img/bam_diff.png}}
  \caption{Kinetikų nesisotinimas nuo šviesos intensyvumo, esant paviršinei sugerčiai}
  \label{fig:nosatur}
\end{figure}

Pav. \ref{fig:nosatur} pavaizduotos dvi kinetikų šeimos, suskaičiuotos esant difuzijai ir be jos. Tikėtasi didinant pradinį krūvininkų kiekį pastebėti kinetikos įsisotinimą į vertę, kuri proporcinga rekombinacijos koeficiento santykiui su Lanževeno rekombinacijos koeficientu \(\frac{B}{B_L} = 1\). Tačiau esant paviršinei sugerčiai \(\alpha d = 100\) tokios soties pamatyti nepavyko. Manome, jog pakeitus programos erdvinio sudalinimą į skirtingo storio narvelius, bus galima suskaičiuoti šias kinetikas, su ypatingai didelėmis šviesos intensyvumo vertėmis.