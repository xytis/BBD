\section{Problemos}

Praeituose darbuose nustatyta, jog esant paviršinei šviesos sugerčiai $(\alpha d > 1)$ sudaromi pakankami krūvininkų pasiskirstymo gradientai, kad modeliuojamose photo-CELIV kinetikose matytųsi difuzinė srovės komponentė.
[Paveiksliukas]

Priklausomai nuo skaičiavimo būdo skiriasi difuzijos įtaka skaičiuojamam rezultatui.

Dėl atsiradusios difuzijos komponentės pakinta šie matavimų rezultatai:

\subsection{Rekombincijos matavimai}

Skaičiavimai atlikti naudojant bimolekulinės Lanževeno rekombinacijos atvejį. Suskaičiavus srovės kinetikas esant skirtingoms užlaikymo trukmėms pagal kinetikos maksimumo vertės kitimą perskaičiuotas rekombinacijos koeficientas. Jis palygintas su pradinių skaičiavimuose naudotu koeficientu.

[Ar reikia formulės?]

[Paveiksliukas] 

Pastebėta, jog atsiranda neatitinkimai tarp realaus (nustatyto) rekombinacijos koeficiento ir apskaičiuoto (virtualiai išmatuoto) koeficiento.

\subsection{Judrio matavimai}

Pagal [čia] aptartą būdą skaičiuojant krūvininkų judrį iš srovės kinetikų gaunama tokia krūvinikų judrio priklausomybė nuo užlaikymo laiko.
[Paveiksliukas]

Anksčiau panašūs rezultatai buvo aiškinami krūvininkų prilipimu. Tačiau šioje simuliacijoje krūvininkų prilipimas nėra įskaitomas, taigi egzistuoja kitos priežastys šiam kitimui.

\subsection{Skaičiavimai} 

Šitiem dalykams stebėti mes suskaičiavom tai: ... ?
