\section*{Santrauka}
Pastaruoju metu sparčiai vystoma fotogeneruotų krūvininkų ištraukimo tiesiškai kylančia įtampa metodika (photo-CELIV), leidžianti tirti plonasluoksnių organinių darinių elektrines savybes. Šiame darbe analizuojama photo-CELIV rezultatų patikimumas, kai bandinyje pasireiškia difuzija. Tiriamas rekombinacijos spartos ir krūvininkų judrio matavimo rezultatų pokytis. Pademonstruojamos paklaidos atsirandančios dėl difuzijos skaičiuojant rekombinacijos koeficientą pagal photo-CELIV srovės kinetikos maksimumo soties vertę ir parodoma difuzijos įtaka judrio vertės skaičiavimams pagal photo-CELIV srovės kinetikos maksimumo padėtį.

\section*{Summary} 
Lately much attention is attracted to photogenerated charge carrier extraction by linearly increasing voltage (photo-CELIV) technique which allows the study of electrical properties in thin-film organic samples. This work aims to analyse photo-CELIV application in measurements of carrier mobility and recombination coeficient when diffusion is observed in the sample. The alteration of measurement results are studied. This work shows the error in recombination coeficient and carrier mobility calculations from saturated photo-CELIV current transients.