\section{Svarbiausi rezultatai ir išvados}
\begin{itemize}
\item Difuzija padidina išmatuotą rekombinacijos koeficientą bandinyje visais tirtais atvejais, jei matavimui naudojama photo-CELIV srovės kinetikos soties vertė. Pokytis gali siekti 20-50\% priklausomai nuo krūvininkų sugerties profilio.

\item Pastebėjome, kad krūvininkų judrio matavimo rezultatai iš photo-CELIV kinetikos maksimumo padėties priklauso nuo užlaikymo trukmės. Šią priklausomybę gali sąlygoti krūvininkų kiekio kitimas \cite{juška:155202}. Tačiau įskaičius difuziją, esant $t_{delay} < t_{tr}$ judrio vertė papildomai pakinta iki 2 kartų, priklausomai nuo pradinio pasiskirstymo.

\end{itemize} 
