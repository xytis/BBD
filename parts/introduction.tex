\section{Įvadas}

\paragraph*{Tikslas}
\begin{itemize}
\item Kodėl žmonėms įdomus photo-CELIV?
\item Ką mas pastebėjome nagrinėdami ją anksčiau?
\item Ką reikia padaryti, kad geriau suprastume difuzijos įtaką?
\end{itemize}

Pastaruoju metu sparčiai vystoma nauja mokslo šaka, nagrinėjanti puslaidininkinius organinius darinius. Mokslininkai, kurdami naujas organines medžiagas, didelį dėmesį skiria elektrinių savybių gerinimui.

Organinių medžiagų tyrimui pastaruoju metu itin dažnai naudojama fotogeneruotų krūvininkų ištraukimo tiesine įtampa (photo-CELIV) metodika \cite{juška:155202}. Deja, vienos iš svarbiausių medžiagos charakteristikų, krūvininkų judrio, nustatymui naudojami teoriniai sąryšiai sukurti remiantis modeliais, kuriuose fizikiniai reiškiniai atskirti vienas nuo kito \cite{langevin}. Tuo tarpu organinėse medžiagose išmatuotas krūvininkų judris gali priklausyti nuo elektrinio lauko, prilipimo lygmenų, krūvininkų rekombinacijos ar difuzijos. Akivaizdu, jog tokiais atvejais tikslaus analitinio modelio sudarymas gali būti itin komplikuotas arba netgi neįmanomas, taigi tenka remtis skaitmeniniu modeliavimu.

Šis darbas yra anksčiau rašytuose darbuose \cite{vytis:kursinis, juška:155202} nagrinėtos difuzijos įtakos krūvininkų pernašai ir eksperimentiniams rezultatams tęsinys. Pastaruosiuose darbuose nagrinėta fotogeneruotų krūvininkų ištraukimo tiesine įtampa metodika (photo-CELIV), taigi neatsitiktinai šiame darbe bus nagrinėjama paviršiuje koncentruotų krūvininkų dinamika. Dėmesys sutelktas į difuzijos įtaką rekombinacijai ir iš to sekančius rezultatus.
Skaičiavimai atlikti su autoriaus sukurta programa \cite{vytis:openreadings2010}. Platesnis programos aprašymas ir programos kodas patalpintas GitHub svetainėje \cite{doi@github}.

\paragraph{Darbo tikslai}
\begin{itemize}
\item Naudojant skaitmeninį modeliavimą nustatyti eksperimentines sąlygas, kurioms esant negalima atmesti difuzijos įtakos gautiems rezultatams.
\item Naudojant skaitmeninį modeliavimą paaiškinti difuzijos įtaką rekombinacijai.
\end{itemize}
