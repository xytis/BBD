\section{Įvadas}

Pastaruoju metu sparčiai vystoma nauja mokslo šaka, nagrinėjanti puslaidininkinius organinius darinius. Kuriamos naujos organinės medžiagos, ypač didelis dėmesys skiriamas jų elektrinėms savybėms. Šių medžiagų tyrimui itin dažnai naudojama fotogeneruotų krūvininkų ištraukimo tiesine įtampa (photo-CELIV) metodika \cite{juška:155202}. Kadangi ši metodika turi taikymo ribas, kurios dar nėra pilnai ištirtos, ji gali būti naudojama netinkamai bei duoti klaidingus rezultatus.

Vienos iš svarbiausių medžiagos charakteristikų, krūvininkų judrio, nustatymui gali būti naudojama photo-CELIV kinetikos maksimumo padėtis. Tačiau pastebėta, jog kinetikos maksimumą dažnai keičia eksperimento sąlygos. Ankstesniame darbe įvertinta krūvininkų žadinimui naudojamos šviesos intensyvumo įtaka kinetikos maksimumui \cite{juška:4946} ir įvestas skaitmeniškai suskaičiuotas pataisos koeficientas. Analizuojant pastarąją problemą sudarytas analitinis modelis buvo neišsprendžiamas, taigi teko remtis skaitmeniniu modeliavimu. 

Šis darbas yra anksčiau rašytame darbe \cite{vytis:kursinis} nagrinėtos difuzijos įtakos krūvininkų pernašai ir eksperimentiniams rezultatams tęsinys. Dėmesį sutelkėme į difuzijos įtaką rekombinacijai ir iš to išplaukiančius rezultatus. Skaičiavimai atlikti su autoriaus sukurta programa \cite{vytis:openreadings2010}. 

\paragraph{Darbo tikslai}
\begin{itemize}
\item Naudojant skaitmeninį modeliavimą nustatyti eksperimentines sąlygas, kurioms esant negalima atmesti difuzijos įtakos gautiems rezultatams.
\item Naudojant skaitmeninį modeliavimą pademonstruoti ir paaiškinti difuzijos įtaką rekombinacijos koeficiento matavimams.
\item Naudojant skaitmeninį modeliavimą patikrinti krūvininkų judrio skaičiavimų rezultatų validumą.
\end{itemize}
