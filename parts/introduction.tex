\section{Įvadas}

\paragraph*{Tikslas}
\begin{itemize}
\item Kodėl žmonėms įdomus photo-CELIV?
\item Ką mas pastebėjome nagrinėdami ją anksčiau?
\item Ką reikia padaryti, kad geriau suprastume difuzijos įtaką?
\end{itemize}

Pastaruoju metu sparčiai vystoma nauja mokslo šaka, nagrinėjanti puslaidininkinius organinius darinius. Mokslininkai, kurdami naujas organines medžiagas, didelį dėmesį skiria elektrinių savybių gerinimui.

Elektrinių medžiagos savybių, tokių kaip krūvininkų judris, nustatymui naudojami teoriniai sąryšiai su išmatuojamomis charakteristikomis, pavyzdžiui krūvininkų lėkio trukme. Dažnai, pastarieji sąryšiai būna sukurti remiantis teoriniais modeliais, kuriuose fizikiniai reiškiniai gali būti atskiriti vienas nuo kito. Pavyzdžiui kietojo kūno fizikoje krūvininkų judris, paprasčiausiu atveju, nepriklauso nuo medžiagoje esančio elektrino lauko \cite{langevin}. Tačiau organinėse medžiagose elektrines savybes nulemiantys reiškiniai dažnai būna persipynę tarpusavyje, pavyzdžiui: krūvininkų judrio priklausomybė nuo elektrinio lauko, krūvininkų prilipimo reiškiniai, krūvininkų rekombinacija ir difuzija. Tokiais atvejais analitinio modelio sudarymas gali būti itin komplikuotas arba netgi neįmanomas, taigi tenka remtis skaitmeniniu modeliavimu. 


Šis darbas yra anksčiau rašytuose darbuose \cite{vytis:kursinis, juška:155202}
nagrinėtos difuzijos įtakos krūvininkų pernašai ir eksperimentiniams
rezultatams tęsinys. Pastarieji darbai nagrinėjo fotogeneruotų krūvininkų ištraukimo tiesine įtampa metodiką (photo-CELIV), taigi neatsitiktinai šiame darbe bus nagrinėjama paviršiuje koncentruotų krūvininkų dinamika. Dėmesys sutelktas į difuzijos įtaką
rekombinacijai ir iš to sekančius rezultatus.
Skaičiavimai atlikti su autoriaus sukurta programa
\cite{vytis:openreadings2010}. Platesnis programos aprašymas ir programos kodas patalpintas GitHub svetainėje \cite{doi@github}.

\paragraph{Darbo tikslai}
\begin{itemize}
\item Naudojant skaitmeninį modeliavimą nustatyti eksperimentines sąlygas, kurioms esant negalima atmesti difuzijos įtakos gautiems rezultatams.
\item Naudojant skaitmeninį modeliavimą paaiškinti difuzijos įtaką rekombinacijai.
\end{itemize}
